\documentclass{article}

\usepackage{physics}
\usepackage{url}
\usepackage{graphicx}
\usepackage{geometry}
\usepackage{hyperref}

\begin{document}

\section{Referee 1}

\begin{enumerate}
\item \textbf{Fig. 1(c) consists of a sketch of the qubit frequencies, which is
	fine, but the authors seem to refer to it in the caption and the main
	text as if it is a measurement of the actual flux modulations for the
	qubits. Simply inserting “Sketch of” or “Schematic of” at the start of
	the caption for 1(c) would solve the issue.} 

Yes, we agree, inserted "sketch of" in Fig.1c

\item \textbf{For the coupling strength J in Table I, the authors should describe
	how J was determined. This could be explained in an appendix.}

This is a good question. We have determined $J$ experimentally while matching the measured and simulated spectra by looking at the size of the avoided crossing between 11/2 and 20/2; it is clearly visible in Fig.2a,b. Its size is $(2\sqrt{2} J)/2$ where the $\sqrt{2}$ comes from the matrix element $\bra{20}\hat H_{int}\ket{11}$ and division by 2 is necessary because it is formed by two-photon processes. The accuracy of this method is not very high (around 10\%), but is enough to prove that the two transmons are indeed coupled to each other. 

We have moved the table to Section II B and added a description for it in the text containing a reference to Section III A where we put the explanation from the first paragraph of this answer.

\item \textbf{In the first paragraph of Sec. II.A, the authors should include a
	definition of the asymmetry parameter d.}

We have included a reference to the original transmon paper and specified that these are the SQUID asymmetries.

\item \textbf{The notation with superscript numbers in Eqs. (3, 4) is a bit
	confusing. It would be better to use a different notation, perhaps
	move the numbers denoting qubits to subscripts? or keep them in
	superscripts but inside of parentheses? Also, the text after Eq. (5)
	mistakenly says that numbers denoting qubits are in subscripts, when
	they are actually superscripts.}

Thank you, all superscripts are now in parentheses.

\item \textbf{In the text just before Eq. (8) — “valid approach until the
	coupling strength J is not too large” — is presumably not what was
	intended. “until” should be perhaps “as long as”. Also, it would help
	to be more quantitative here and explain at what scale of J the
	theoretical approach would break down.}

Thank you, we have clarified the sentence and have explicitly said that $J$ should be much less than $\omega_{1,2}$. More details may be found in the paper referenced there.

\item \textbf{For the spectroscopy measurements in Fig. 2 and 3, the use of the
	term two-tone spectroscopy is a bit confusing...}

Yes, indeed. We have now resolved this ambiguity, two-tone spectroscopy is now only mentioned in the Appendix with clear description of what that means.

\item \textbf{For the experimental description, the authors should include more
	details about how the qubit readout was done. Presumably S21 was
	measured on the center of the readout cavity dip when neither qubit is
	being excited, or perhaps the readout frequency was tuned to be
	somewhere on the side of the resonance for greater response? Simply
	plotting S21 at a particular frequency, rather than extracting the
	readout resonance frequency for each spectroscopy point, relies on the
	dispersive shift of the cavity being less than its linewidth. Perhaps
	that is the case here? These details of the readout procedure should
	be discussed, at least in the Appendix.}

No, indeed our script has been fitting the notch-type $S_{21}$ response at each point and then setting the readout frequency to match the lowest point in the transmission. We have added a note in Appendix E along with the link to the source code of the measurement class defining the logic of the experiment.

\item \textbf{ In Fig. 2(d), the use of only color to distinguish the various
	curves is difficult to follow, particularly if one prints out the
	paper in black and white. Also, I couldn’t find a description of the
	dotted-line curves that are labeled “others” in the legend. It’s not
	clear what these are.}

Agreed. Added different types of dash for the plot, should be more readable now in print; 'others' label is now also clarified.

\item \textbf{9) In the first line of text after Eq. (14) — “Tailor” should be
	“Taylor”.}

Thank you, fixed.

\end{enumerate}

\section{Referee 2}


\begin{enumerate}
	\item \textbf{ In the introduction the authors say their work is "bridging
		superconducting devices with atomic optics". What does that mean?}
	
Thank you, this was indeed not a clear statement. Our message was that our experiment is very similar to what is usually done with natural molecules (we use a straightforward spectroscopic approach and then explain the origins of the observed spectroscopic lines) and that our results may be interesting not only to the scientists in the field of superconducting quantum devices, but to a broader audience studying the phenomena arising in composite quantum systems. 

We will rephrase that sentence to be more clear and to convey exactly the idea above.

	\item \textbf{In the conclusion they say that these effects "can not be observed
		with natural systems". Why is that? The actual origin of the observed
		transverse interaction is not discussed in any detail, nor is its
		strength and the specific relationship and the potential limitations
		of atomic and molecular systems.}
	
	We would like to change the phrasing -- these effects are surely possible to observe in principle in any kind of diatomic system but the conditions for that are rather strict as can be seen from the paper. For example, we need a resonance between transitions of different orders for ``II'' and ``III'', and individual addressability of the atoms to observe effect ``I'' to ensure different Rabi frequencies. 
	
	It seems that for natural molecules that can't be tuned in frequency and whose parts can't be addressed individually due to their size, this may be a rare case. However, it is indeed not impossible.
	
	Considering the origin of the transverse interaction: we have avoided discussing it, firstly, because it has already been very densely covered for transmons in literature, and, secondly, to make the model of the system as simple as possible to make it understandable for a wider audience while retaining all the effects present (nevertheless, we will now discuss it in new Appendix B). For natural molecules, the interaction Hamiltonian will certainly be different. However, as long as it ensures hybridization of states and, therefore, non-zero transition matrix elements of the first system driving operator between the states, the reported effects should work exactly the same. We think that for each individual case the calculations will be individual, too, but the general idea above will apply everywhere.
	
	The strength of interaction in our case is moderate, and it certainly does not exceed the coupling in natural molecules. Therefore, it will not be a prohibiting parameter for the emergence of the effects in other coupled systems.
	
	Considering the relationship of our interaction with the real interactions in natural molecules, we can't state anything quantitatively because we are not experts in that field. But as long as we now do not use such a strong wording regarding the ``impossibility'' to observe the described effects, this should not really be necessary!
	
	\item \textbf{ It would help the appeal of the paper a lot if the authors could
	find and mention additional motivation for this quite technical work.
	Why are these nicely looking features interesting? One could think
	about how these interactions and the participating states are
	different from any other ones observed so far?}
	
	As we have been looking through a lot of published works, we are certain that all the processes labelled I (please, see our following answers regarding Filipp2011), II and III are novel and have not been analysed previously. For example, the interplay between transitions 12/3 and 10-02 was not discussed in literature before, just as the splitting of the line 11/2 in feature I. Moreover, there were no examples of high-intensity spectroscopy of two-transmon systems in literature; we think that this regime should definitely be covered and shown to comply with theory just as well as the low-power studies, especially if these effects can be easily reproduced just by a single-tone excitation of the SAM. Finally, we have introduced the concept of self-consistency in the rotating frame that predicts the actual frequencies of Autler-Townes-like sideband transitions when they are caused by the same field that forms the levels participating in the transition. While this idea looks quite clear now, it was completely not obvious in the beginning due to the complexity of the level structure.
	
	We have bolstered the 4th and 5th paragraphs of the introduction with these thoughts and new references to literature, and hope that now the paper presents itself better.
	
	\item \textbf{In general I find the review of previous work quite comprehensive with
	one exception I am aware of, i.e. S. Filipp PRA 83, 063827 (2011),
	which I find relevant in multiple contexts of this work (see also
	below) and the authors should clearly communicate how their work is
	different and which parts are related.}

	Thank you for this thorough and extensive paper. We were not aware of it, and will discuss the differences between our results and theirs below.
	
	We now have also added some new works containing examples of previous spectroscopic studies of two-transmon systems to our list, including Filipp2011
	
	\item \textbf{looking at the design, I see 2 potential reasons for qubit-qubit
		interactions. Direct capacitive coupling and dispersive coupling via
		the resonator. The authors should quantify the two effects and discuss
		what type of interactions are generated in the two cases. Why are
		capacitive effects negligible?}
	
	Thank you for that question, it has helped us to significantly improve our paper. 
	
	We indeed were aware of both these mechanisms, but while designing the chip in the simulations we were only tracking the full $J$ from the numerical diagonalization and did not evaluate explicitly its parts. Moreover, we did not take into account the other modes. 
	
	It turns out that the contributions from these two mechanisms are of different signs for our case, and in overall the system is nearly identical to what is described here: Yan, Fei et al, Physical Review Applied 10.5 (2018), 054062 except for the fact that our coupling resonator has many modes.
	
	Now, in a new Appendix A, we do all the math and find that actually the direct coupling dominates even though the capacitance between the transmons is just 0.3 fF. Moreover, we use there the recently published results to find the relevant cutoff frequency to find the multimode coupling. Please, see Appendix A for our results.
	
	
	\item \textbf{even just taking into account the dispersive interaction, using the
	nonlinear oscillator basis hides in some ways an intuitive
	understanding of how these systems interact as would be obtained in
	the qubit (sigma) basis (transverse flip flop) and loosing information
	about the symmetries. I would find it important to at least mention
	explicitly the state that are formed in this basis and the type of
	interaction they get (and contrast it to e.g. the direct capacitive).
	}
	
	Since now in Appendix A we explicitly follow Yan, Fei et al, Physical Review Applied 10.5 (2018), 054062, we should not be loosing any symmetries. The non-linear oscillator basis matches 
	with the linear oscillator basis up to the factor of $i$ and thus just a global phase. This can be seen from the representation of the charge operator:
	\[
	\hat n \propto \hat a^\dag + a\  \text{ in the transmon eigenbasis or}\ \hat n \propto i(\hat a^\dag - a)\ \text{in the oscillator basis}  
	\]
	
	Thus, in RWA we recover exactly the same meaning of $\hat H_{int}$: the tunnelling of photons between oscillators which generalizes the tunnelling of an excitation back and forth between two-level systems. In RWA,  $\hat H_{int}$ are identical in both representations, and truncated to two levels fully recover the sigma operators. The counter-RWA parts, in contrast, have opposite signs, but this only matters for the dispersive coupling and is cleared with the right choice of the Schrieffer-Wolff transformation.
	
	To defend our choice further, we want to highlight that using the multilevel basis is a widely adopted formalism now for describing transmon arrays with the Bose-Hubbard model, for example ``Hacohen-Gourgy, Shay, et al. Physical review letters 115.24 (2015): 240501'' and  ``Neill, Charles, et al. Science 360.6385 (2018): 195-199)''.
	
	
	\item \textbf{the qubit to resonator coupling strength is not mentioned and it
		would be good to see that the observe J coupling is actually in
		agreement with the expected J coupling due to the dispersive
		interaction with the resonator (potentially including higher resonator
		modes). One should be able to back out the gs from the dispersive
		shifts and detuning quite accurately.}
	
	This is now done thoroughly in Appendix A, thank you. Indeed, we have found interesting results and probably even new directions for us to work on from this.

	\item \textbf{J(omega) is mentioned to be approximately constant but its physics
	(or the physics of the origin of J) and the experiments investigating
	it are not mentioned (multi-mode aspects, cut off frequencies, change
		of matrix elements, symmetries and selection rules, ...). In my
		opinion the authors should mention that this has been done. Ideally
		they should also quantify how constant J is for their range of qubit
		frequencies and if it agrees with the observed dispersive shifts (and
		state the two g1 and g2).}
	
	We have now covered the question of origins of $J$ extensively in Appendix A. Since our frequency span of the transmons is around 10\%, thus the change of $g_1, g_2$ should be around 5\% due to their square root dependence. However, since the transmons move in opposite directions, the change is of order 2.5\%, which is comparable to our linewidths and thus $J$ should be acceptable as constant.
	
	Multimode aspects are now covered, especially the cut-off problem. The change of the matrix elements for transmons are very well described by the square root dependence on their frequency. Using the location of the dark state in the lower branch of the avoided crossing between 01 and 10 (see below) and the positive symmetry of the drive we confirm the positive sign of $J$ for our case.
	
	\item \textbf{The discussion of feature I, which as far as I can tell is the same
		as what has been studied in Filipp2011, seems to be at odds in some
		aspects (see also below).}
	
	Actually, no, by ``feature I'' we mean a different effect: the ``avoided crossing'', as we have called it, in I is formed by the transitions 11/2 and 01 (or 10 depending on the interplay between $\Omega_2$ and $\Omega_2$). Note the quotes that we use because I is not an avoided crossing in the usual sense. If we look closely at Fig. 4 of our paper, we see that the line 11/2 bends when approaching the intersection of 10, 01 from one side and then finally gets absorbed by one of them. It then reappears on the other side, emerging from the same line it was absorbed by before. That deformation and apparent splitting of the line 11/2 is what we call feature I, and we apologize for the ambiguity.
	
	In sharp contrast, in Filipp2011 the avoided crossing is between 01 and 10, while 11/2 is passing straight through the center of the anticrossing. We have observed the same effect for lower drive powers, which can be seen in \autoref{fig:0110xx} (data from the second cooldown, driving through the resonator), and in \autoref{fig:dark_state_exp} (first cooldown, driving through the charge line). 
	Moreover, the anticrossing from Filipp2011 is reproduced by the unperturbed model, as can be seen in Fig. 2(d) of our paper, while feature I is not.
	
	In Fig. 2 of our paper, the avoided crossing from \autoref{fig:0110xx} is smaller than the linewidths of 01 and 10 and can not be resolved, just as in Fig. 4 of our paper.
	
	\begin{figure}
		\centering
		\includegraphics[width=.7\linewidth]{{Pictures/10-01XX}}
		\caption{The avoided crossing between 01 and 10 for our system at low power, reproducing the result from Filipp2011 (the transmons are driven through the resonator here, in contrast to what is presented in Fig. 2 of our paper). The lines are the numerical diagonalization result. The dark state, however, is not visible in this scan even though we see it when driving the SAM through the antenna for similar power and resolution.}
		\label{fig:0110xx}
	\end{figure}


\begin{figure}
	\includegraphics[width=.49\linewidth]{Pictures/dark_state_exp}
	\includegraphics[width=.49\linewidth]{Pictures/dark_state_exp_2}
	\caption{Two of the experimental scans which has captured the dark state at lower power (the driving is via the microwave antenna). The visible asymmetry is consistent with the drive asymmetry.} 
	\label{fig:dark_state_exp}	
	
\end{figure}
	
	So, due to the different nature of the effect that we call feature I, our results do not contradict any previous well established results. Please see also \autoref{fig:dark_state} for clarification of what happens with the transition 11/2 when the drive power is increased.
	
	
	\begin{figure}
		\includegraphics[width=\linewidth]{Pictures/dark_state}
		\caption{The avoided crossing between 01 and 10 for increasing drive strengths $\Omega_1 = 2\Omega_2$. As one can see, the dark state is shifted to the left when the asymmetry of the eigenstates matches the drive asymmetry, by vanishes gradually when the drive is increased. At the largest power, the avoided crossing between 01 and 10 disappears, and the lines look simply passing through each other with no deformation. Instead, the line 11/2 is now deformed peculiarly. This is what we call feature I. Note also the population inversion that occurs here at high powers.} 
		\label{fig:dark_state}	
	\end{figure}
	
	\item \textbf{The symmetry of the drive seems to be not important in these new
	experiments and there is no observation of dark states in the
	spectroscopic lines. Why is this? Maybe because the transmons are
	located on the same end of the resonator? Or because they are driven
	via the external (common) charge line?}
	
	First of all, yes, the qubits are driven through a common charge line. From what is covered in literature, the dark state is expected to appear only for the intersection between 01 and 10 when there are certain conditions on the symmetry of the drive.
	

 	In \autoref{fig:0110xx} and \autoref{fig:dark_state_exp}, we have shown our corresponding data, and in \autoref{fig:dark_state} we reproduce this dark state in a numerical simulation with the same model that was used in our paper. One can see that the dark state is indeed present, but can't be seen for high powers because even the smallest matrix element of the drive already saturates the qubit in the steady state (drives it into the state with zero Z projection).
	In \autoref{fig:0110xx} and similar scans that are done through the resonator, we do not see that dark state, but the scans via the charge line do reveal it exactly where it is expected to appear; we provide two scans in \autoref{fig:dark_state_exp}. We can't explain quantitatively now why the dark state is not visible in the first scan. Most probably, the drive is composed both of the direct (over the resonator) and the dispersive (through the resonator) parts with different asymmetries, therefore no superposition is dark with respect to such drive. However, the 11/2 transition is still visible the best only in the area of the avoided crossing so the logic from Filipp2011, Section V, should imply that there is indeed a suppression of one of the transition amplitudes. But since this avoided crossing is not what we are trying to explain in our paper, we believe that the investigation of this fact should be postponed, especially as long as the charge line behaves as expected.
	
	Another point we would like to highlight answering this question is that, we in fact do see one dark state which is in the lower branch of the avoided crossing named as feature 3 (Arabic). It is particularly fascinating as it is the three-photon dark state, and appears not shifted and thus insensitive to the symmetry of the drive (we did not thoroughly check that). Unfortunately, we could not explain it so far (even though it is reproduced in the numerical simulation), but we still mention it in Section III B, paragraph \textit{c}.
	
	We will add another picture of the spectrum (from the second cooldown) in the Appendix at even a higher power where it is visible even better.
	
	\item \textbf{Are there any other differences of this new work compared to the
		previous one that I am missing?}
	
	We would say that we do not intersect with Filipp2011 in any way concerning the results, and the only common object of the study is the system itself (except that the we have two coupling mechanisms but we did not use that fact in our theoretical modelling).
	
	The results of Filipp2011 are (transmons are treated as qubits): multimode model for coupling strength including spurious resonances, the explanation of the dark state origins and location in the avoided crossing 01-10, two-photon transition 00 - 11 explanation.
	
	In contrast, we are from the beginning treating transmons as multilevel systems (qutrits is enough to capture the effects). In the main text, we do not study the coupling mechanisms, do not present the avoided crossing between 01 and 10 in the experimental data, as it is not visible for our drive powers; consequently and we do not state anything about the symmetry and location of its dark state. 
	
	Moreover, now we have included the multimode analysis for the coupling and cut-offs, which we believe is very valuable to the field, and should be further investigated given new theoretical results.
	
	From said above it should be clear that all of our results were not reported before. We report not just the multiphoton transitions that can occur in a two-qubit system, but we look at those occuring in a two-qutrit system. And what is more important, we do not just show them, but demonstrate how the dressing by intense driving even just with a single tone affects them, modifying their frequencies and qualitative appearance according to standard quantum mechanical laws. Finally, we present accurate steady-state modelling for qutrits which was not done before in Filipp2011 for two qubits.
	
	\item \textbf{Feature I in Fig2d (right hand side) clearly shows an anti-crossing
		(also the lines which according to the legend represent the
		unperturbed Hamiltonian) but later on it is said that feature I does
		not show an avoided level crossing if either the drives are the same
		or the Hamiltonian is unperturbed. Maybe the authors could clarify
		this.}
	
	We apologize again for this ambiguity. We name as feature I the apparent splitting of the transition 11/2 and the deviation of its frequency from the unperturbed Hamiltonian prediction while the avoided crossing 01 vs 10 is vanished and already not visible at such power. Looking closely at the color data in Fig. 2d of our paper, especially left hand side, one can not 
	resolve an anticrossing formed by the lines 01, 10 drawn as the model lines in the right. The only effect that is seen is that 11/2 is being bent and split, and that's exactly what we are interested in. So the problem is that we did not state explicitly the difference between feature I and the avoided crossing of 01, 10, and now we are doing that in the beginning of section III A.
	
	We will change the wording considering feature I. We will now call it a ``splitting of the spectral line 11/2'' to avoid confusion with the effect well studied in Filipp2011.
	
	``but later on it is said that feature I does
	not show an avoided level crossing if either the drives are the same
	or the Hamiltonian is unperturbed''
	
	We couldn't find the exact quote in our paper, but we can clarify this: yes, the splitting of 11/2 is not observed when the drive is symmetric, though from Fig. 4 of our paper we can see that its frequency is somewhat shifted upwards (we could not explain that). And if the Hamiltonian is unperturbed (no driving), then we only have the eigenlevels which of course yield continuous 11/2 line, as shown in Fig. 2 d of our paper. Additionally, if the Hamiltonian has no coupling term, 11/2 will not be visible, so feature I will not be present as well.
	
	\item \textbf{The discussion of feature I seems also to be inconsistent with what
		has already been observed in Filipp2011 where the qubits were driven
		via the resonator with approximately equal strength. Maybe the authors
		could clarify this point as well.}
	
	As I hope is clear from above, this is not the case, we just should have been more explicit in the paper stating the difference between feature I and the standard anticrossing located at the same point.
	
	\item \textbf{Specifically, why does the avoided crossing vanish for equal drive
		strength?}
	
	Talking about the feature I how we mean it (splitting of 11/2): this is explained it in the section III D, paragraph b, using Fig. 6 of our paper.
	
	The usual avoided crossing covered in Filipp2011 truly does not depend on the drive since it is not a consequence of light dressing, but one can't resolve it for our powers and linewidths in Fig. 4 of our paper.
	
	
	\item \textbf{Why is there no dark state (in feature I)?}
	
	We don't know why there are no selection rules in that case (how we mean it). But the numerical simulation supports that.
	
	Considering the usual avoided crossing, we have shown above that we can see it, and it is consistent with Filipp2011.
	
	\item \textbf{- the design looks beautiful and symmetric. Do the authors have a
	suspicion why the charge line couple twice as strong to one qubit than
	the other? Maybe there is a capacitive of even inductive interaction
	at play as well? If so, how would that modify the assumptions about
	the J coupling interaction?}
	
	Actually, the design is a bit asymmetric, as can be seen from Fig. 1a of our paper, the charge line goes closer to the higher frequency transmon (with larger Josephson junctions). Even though the line goes relatively far from it, and should have bean screened by airbridges, it still seems to couple stronger to the closer transmon. The coupling is still capacitive. 
	
	We do not suspect inductive interaction since transmons are far less sensitive to that even for asymmetric SQUIDs (see the original paper, Koch2007), and we bring an open end of the line to the SAM, so the magnetic fields should be suppressed there.
	
	All that should not modify the assumptions about
	the $J$ coupling interaction because the resonator itself is symmetric, and we do not find any differences in the coupling capacitances to is both experimentally and from FEM modelling (see Appendix A.)
	
	\item \textbf{- at which flux/qubit frequencies were the T1 and T2 times extracted?
		Were they assumed to be frequency independent in the numerical model?}
	
	Yes, at the sweet spots, we mentioned that in the table. We assumed frequency-independent parameters in the model, which worked well enough because we have very strong driving and the linewidths are mostly larger than the dissipation rates.
	
	\item \textbf{- it is mentioned that the parameters changed due to a second
		cooldown. How and why did they change? What are the new values?}
	
	\begin{figure}
		\centering
		\includegraphics[width=\linewidth]{Pictures/II-TTS}
		\caption{The spectrum of the SAM in the second cooldown.}
		\label{fig:ii-tts}
	\end{figure}
	
	
	This is a valid question, please find the new spectrum equivalent to Fig.2(a) of our paper in \autoref{fig:ii-tts}. The Josephson energies and sweet spot locations are shifting between cooldowns. While sweet spots can be readjusted, the frequency locations of the maxima and minima of the qubits can not, therefore we had a problem that feature III became too deformed with the new configuration of the transmon frequencies, and Feature I in now interfering with the spurious resonance. 
	
	We chose not to reestimate the parameters for the numerical model because it is a cumbersome process and does not affect our final conclusions.
	
	\newpage
	
	\item \textbf{
		- it is mentioned that the change of the spectrum is of topological
		nature. What does that mean in this context?}
	
	We mean that the resulting configurations in Fig. 4 of our paper (row 1 vs. 2) can not be transformed into each other by a continuous deformation; this is because the lines that participate are different.
	
\end{enumerate}


\end{document}